\documentclass{article}
\usepackage[backend=biber, style=numeric]{biblatex}
\usepackage{hyperref} % must be last import

\addbibresource{produktrapport.bib}

\newcommand{\projectname}{Decentraliseret dataudveksling i notesapplikation}
\newcommand{\student}{Marcus Haukelid Larsen}
\newcommand{\supervisor}{?}

\begin{document}

\pdfbookmark[1]{Forside}{frontpage}

\begin{center}
  {\Huge \textbf{\projectname}}

  \vspace*{\fill}

  {\Large \textbf{\student}}

  \vspace*{\fill}

  {\large \textbf{Produktrapport}}
\end{center}

\thispagestyle{empty}

\newpage
\begin{center}  
  {\large \textbf{Elev:} \student}
  \vspace*{\fill}

  {\large \textbf{Projektnavn:} \projectname}
  \vspace*{\fill}

  {\large \textbf{Uddannelsessted:} Techcollege, Struervej 70, 9220 Aalborg Øst, Denmark}
  \vspace*{\fill}

  {\large \textbf{Elevplads:} RTX A/S, Strømmen 6, 9400 Nørresundby, Denmark}
  \vspace*{\fill}

  {\large \textbf{Projektperiode:} 05/02-2026 - 17/03-2026}
  \vspace*{\fill}

  {\large \textbf{Afleveringsdato:} 17/03-2026}
  \vspace*{\fill}

  {\large \textbf{Fremlæggelsesdato:} 20 el. 23/03-2026}
  \vspace*{\fill}

  {\large \textbf{Vejleder:} \supervisor}
  \vspace*{\fill}

  {\large \textbf{Underskrift:}}
  \vspace*{\fill}
  
  \begin{tabular}{p{6cm} p{6cm}}
    \hrulefill & \hrulefill \\
    Vejleder (\supervisor) & Elev (\student)
  \end{tabular}
\end{center}

\thispagestyle{empty}

\newpage
\tableofcontents

\newpage

\section{Læsevejledning}
Denne rapport er struktureret, så læseren først får en introduktion til
projektets kontekst. I indledningen præsenteres baggrunden for casen
og de centrale spørgsmål, som rapporten søger at besvare.

Efter indledningen følger en beskrivelse af kravspecifikationen, hvor
funktionalitet og acceptkriterier præsenteres. Afsnittet
Anvendte teknologier beskriver de teknologier, projektet benytter,
og er opdelt i tre underafsnit: MicroUI, OpenGL ES 3.0 og Tup.

Rapporten afsluttes med en oversigt over projektets overordnede arkitektur
samt en beskrivelse til opsætning af projektet.

\section{Indledning}
Vi lever i en digitaliseret verden, hvor ejerskab i stigende grad erstattes af
midlertidig adgang.

Som følge af magtkoncentration er der opstået flere organiserede bevægelser, der
fremhæver decentraliseret digital infrastruktur som et muligt alternativ.

Decentralisering rejser nye politiske spørgsmål om ansvar, regulering og ulighed.
Fraværet af centrale aktører kan vanskeliggøre både kontrol og retlig beskyttelse.

Rapporten beskriver, hvordan produktet er udarbejdet, hvilke delelementer det
indebærer, og hvilke begrænsninger der er identificeret undervejs.

\section{Casebeskrivelse}
Vi lever i en digitaliseret verden, hvor ejerskab i stigende grad
erstattes af midlertidig adgang. Denne udvikling rejser politiske
spørgsmål om magt, rettigheder og afhængighed i relationen mellem
borgere, stater og globale tech-giganter. Når data fungerer som den
centrale valuta i den digitale økonomi, forskydes magten fra borgeren
til de platforme, der kontrollerer indsamling, behandling og
anvendelse af disse data.

Som reaktion på denne magtkoncentration er der opstået flere
organiserede bevægelser, som fremhæver decentraliseret digital
infrastruktur som et muligt alternativ. I en decentraliseret model
er data og digitale aktiver ikke samlet hos enkelte platforme, men
distribueret på tværs af netværk, hvor brugerne i højere grad 
har kontrol over egne oplysninger og digitale ressourcer.

Decentralisering rejser dog også nye politiske spørgsmål om ansvar,
regulering og ulighed. Fraværet af centrale aktører kan vanskeliggøre
både demokratisk kontrol og retlig beskyttelse, hvilket udfordrer
statens traditionelle rolle som garant for borgernes rettigheder.

\section{Problemformulering}
Hvordan kan decentraliseret dataudveksling i en notesapplikation
bidrage til øget digitalt ejerskab og reducere brugerens afhængighed af
centrale platforme uden at gå på kompromis med tilgængelighed og 
brugervenlighed?

\section{Kravspecifikation}
Formålet med en kravspecifikation er at skabe en entydig forståelse af
produktet, så man kan fastlægge, dokumentere og kommunikere, hvad
produktet skal gøre præcist, konkret og målbart.

\subsection{Funktionalitet}
Nedenstående tabel beskriver den ønskede funktionalitet for produktet
og angiver dermed, om produktet er fuldført. Det giver et konkret
udgangspunkt.

Der benyttes tre kategorier: grænseflade, lagring og udveksling.
Prioritering består af to niveauer: 1 = need-to-have, 2 = nice-to-have.

\begin{table}[ht]
  \centering
  \begin{tabular}{|l|p{9cm}|c|}
    \hline
    \textbf{Krav} & \textbf{Beskrivelse} & \textbf{Prioritet} \\
    \hline
    G1 & Grænsefladen skal vise en oversigt over alle noter & 1 \\
    \hline
    G2 & Grænsefladen skal kunne oprette nye noter fra oversigten & 1 \\
    \hline
    G3 & Grænsefladen skal kunne tilføje synkroniseringsenheder & 1 \\
    \hline
    G4 & Grænsefladen skal kunne sortere noter i oversigten & 2 \\
    \hline
    L1 & Noter skal lagres persistent & 1 \\
    \hline
    L2 & Noter skal automatisk gemmes & 1 \\
    \hline
    L3 & Konfigurerede synkroniseringsenheder skal lagres persistent & 1 \\
    \hline
    U1 & Alle noter i oversigten skal synkroniseres med de konfigurerede
         synkroniseringsenheder & 1 \\
    \hline
    U2 & Systemet skal håndtere offline-noter og synkronisere,
         når netværk er tilgængeligt & 2 \\
    \hline
    U3 & Systemet skal håndtere konflikter ved samtidige ændringer & 2 \\
    \hline
  \end{tabular}
  \caption{Krav og prioritet}
  \label{tab:krav}
\end{table}

\subsection{Acceptkriterier}
Nedenstående tabel beskriver konkrete og præcise kriterier for, at
produktet opfylder de ovenstående krav, og dermed verificerer, at
produktet er fuldført og virker efter hensigten.

\begin{table}[ht]
  \centering
  \begin{tabular}{|l|p{8.5cm}|c|}
    \hline
    \textbf{Krav} & \textbf{Acceptkriterier} & \textbf{Prioritet} \\
    \hline
    G1 & Oversigten viser alle noter; listen kan scrolle;
         ingen fejl ved mere end 1000 noter. & 1 \\
    \hline
    G2 & Bruger kan oprette ny note fra oversigten via tydelig knap;
         note gemmes og vises umiddelbart. & 1 \\
    \hline
    G3 & Bruger kan tilføje synkroniseringsenhed via siden "Synkronisering";
         enhed valideres, vises i enhedsliste og gemmes persist. & 1 \\
    \hline
    L1 & Noter gemmes persist mellem sessioner og
         er tilgængelige efter genstart. & 1 \\
    \hline
    L2 & Ændringer gemmes automatisk ved fokusændring eller
         inden for få sekunder efter redigering. & 1 \\
    \hline
    L3 & Konfigurerede synkroniseringsenheder er persistente og
         tilgængelige efter genstart. & 1 \\
    \hline
    U1 & Alle noter i oversigten synkroniseres til alle synlige,
         konfigurerede enheder; sync-status vises. & 1 \\
    \hline
    U2 & Lokale ændringer accepteres offline;
         automatisk sync køres ved netværksgenoprettelse uden datatab. & 2 \\
    \hline
    U3 & Konflikter ved samtidige ændringer detekteres;
         bruger tilbydes valg (behold lokal, behold ekstern, merge)
         før overskrivning. & 2 \\
    \hline
  \end{tabular}
  \caption{Acceptkriterier for kravene}
  \label{tab:acceptkriterier}
\end{table}

\section{Anvendte teknologier}

\subsection{MicroUI}
MicroUI er et lille, bærbart immediate-mode layoutsystem skrevet i ANSI C,
designet til indlejrede og ressourcerestrikterede systemer.

Det er meget lille (ca. 1100 linjer ANSI C) og arbejder ud fra en kontekst,
bestående af et forudallokeret hukommelsesområde, så der ikke foretages
dynamisk allokering under drift. Biblioteket indeholder minimale, indbyggede
kontroller: vindue, scrollbart panel, knap, slider, tekstfelt, label,
checkbox og wordwrap-tekst. Det er renderer-agnostisk og kan anvendes med
enhver renderingsmotor, der kan tegne rektangler og tekst. Designet er
enkelt og udvidelsesvenligt, hvilket gør det nemt at tilføje
brugerdefinerede kontroller. Layoutstyringen er letvægtig og enkel at
bruge.

MicroUI passer til indlejrede systemer, og projekter,
der ønsker tæt kontrol over hukommelse og afhængigheder.

\subsection{OpenGL ES 3.0}
OpenGL for Embedded Systems (OpenGL ES eller GLES) er en udgave af
OpenGL API'et til rendering af 2D‑ og 3D‑grafik, typisk
hardware accelereret af et grafikkort. Det bruges primært til
grafisk fremstilling i applikationer som spil og andre grafiktunge
systemer.

OpenGL ES er designet til indlejrede platforme — herunder smartphones,
tablets og spilkonsoller — men anvendes også i webbrowsere via WebGL
og kan køre native på stationære computere. Det er et af de mest
udbredte 3D grafik API'er i historien.

API'et er tvær platform. OpenGL ES forvaltes af
non profit konsortiet Khronos Group, som også står bag Vulkan — et
nyere API rettet mod enklere, højtydende drivere til både mobile og
stationære enheder.

\subsection{Tup}
Tup er et filbaseret byggesystem til Linux, macOS og Windows. Det
modtager en liste over filændringer og en retningsbestemt acyklisk graf
(DAG) og bearbejder grafen for at udføre de nødvendige kommandoer til
opdatering af afhængige filer. Tup udfører opdateringer med lav
overhead ved hjælp af effektive algoritmer, så unødvendigt arbejde
undgås, hvilket giver hurtigere byg og mindre vedligeholdelse af
byggesystemet.

Til beregning af ændringer i DAG'en benytter Tup den såkaldte
beta-algoritme. Algoritmen analyserer indgående filændringer, identificerer
påvirkede afhængigheder og prioriterer minimale opdateringer, så kun
de nødvendige kommandoer køres. Det reducerer arbejde og sikrer hurtigere,
deterministiske byg med lav overhead (se \cite{tup-algorithm}).

\section{Arkitektur}

\section{Opsætning}

\printbibliography[title={Referencer}]

\end{document}
